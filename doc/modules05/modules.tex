%
% File draft.tex
%

\documentclass[11pt]{article}
\usepackage{acl2005}
\usepackage{times}
\usepackage{latexsym}
\setlength\titlebox{6.5cm}    % Expanding the titlebox

\newcommand{\hpsg}{\textsc{hpsg}}
\newcommand{\lkb}{\textsc{lkb}}
\newcommand{\lfg}{\textsc{lfg}}

\title{Rapid Prototyping of Scalable Grammars:\\
       Modular Extensions to a Language-Independent Core}
\author{Emily Bender\\
Department of Linguistics\\
University of Washington\\
Box 354340\\
Seattle, WA 98195-4340, USA\\
{\small {\tt ebender@u.washington.edu}}
\And
Dan Flickinger\\
CSLI, Stanford University\\
Ventura Hall, 220 Panama St\\
Stanford, CA 94305, USA\\
{\small {\tt danf@csli.stanford.edu}}
}

\date{}

\begin{document}
\maketitle
\begin{abstract}
This paper presents a novel approach to simpliyfing the construction of 
precise broad-coverage grammars, employing typologically-motivated modules
as customizable extensions to a language-independent core grammar.  Each
module represents some salient dimension of cross-linguistic variation such
as word order or negation, and presents the grammar developer with simple 
choices that result in automatically generated language-specific software 
that becomes part of the grammar implementation for that language.  The 
paper illustrates this notion of a module for several phenomena, and 
includes an evaluation of the approach against multilingual test data
reflecting interactions among these phenomena.
\end{abstract}

\section{Introduction}
Manual development of precise broad-coverage grammar implementations is a 
labor-intensive undertaking, typically requiring multiple years of work by 
highly trained computational linguists.  Recent efforts toward reducing the
time and level of expertise needed for producing a new grammar have
focused either on adapting an existing grammar of a related language 
(ref ParGram Jpns-Krn), or on identifying a set of language-independent 
grammar constraints to which language-specific constraints can be 
added~\cite{Ben:Fli:Oe:02}.  Neither implementation approach has benefited 
from the substantial theoretical work on language typology, which characterizes
linguistic phenomena and the varied mechanisms that languages employ to 
realize them.  In this paper we report on an elaboration of the grammar
matrix approach in which phenomenon-specific modules encode the dimensions
of variation, and present the grammar developer with simple choices that
determine this language's type for each phenomenon. Each module then
automatically generates the necessary language-specific constraints as
extensions to the Matrix grammar files, enabling the linguist to produce
a working prototype grammar with very little effort or computational
expertise.  We ilustrate this concept of a typology-based module for
several phenomena including basic word order, sentence negation, and
yes-no questions, and we conclude with an evaluation of the resulting
grammar implementations for a variety of languages.

\section{The Grammar Matrix}

The past decade has seen the development of wide-coverage implemented
grammars representing deep linguistic analysis of several languages in
several frameworks, including Head-Driven Phrase Structure Grammar
(\hpsg), Lexical-Functional Grammar (\lfg), and Lexicalized Tree
Adjoining Grammar ({\sc ltag}). In , the most extensive grammars
are those of English \cite{Flickinger:00}, German \cite{Mue:Kap:00},
and Japanese \cite{Siegel:00,Siegel:Bender:02}.  As part of a
multi-lingual grammar engineering effort, we are developing a `grammar
matrix' or starter-kit, distilling the wisdom of existing grammars and
codifying and documenting it in a form that can be used as the basis
for new grammars.

The main goals of the project are: (i) consistent with other work in
\hpsg, to develop in detail semantic representations and in particular
the syntax-semantics interface; (ii) to represent generalizations
across linguistic objects and across languages; and (iii) through the
richness of the matrix and its built-in links with
the \lkb\ grammar engineering environment \cite{Copestake:02}, to
allow for extremely quick start-up as the matrix is applied to new
languages.

The first, preliminary version of the grammar matrix~\cite{Ben:Fli:Oe:02}
consisted of types defining the basic feature geometry and technical
devices (e.g., for list manipulation), types associated with Minimal
Recursion Semantics (see, e.g., \cite{Cop:Las:Fli:01}) types for
general classes of rules (including derivational and inflectional
lexical rules, unary and binary phrase structure rules, headed and
non-headed rules, and head-initial and head-final rules), and types
for basic constructions such as head-complement, head-specifier,
head-subject, head-filler, and head-modifier rules, as well as 
coordination and more specialized classes of constructions, such as relative
clauses and noun-noun compounding.  In addition, the preliminary
version of the grammar matrix included configuration and parameter
files for the \lkb.  The current released version of the matrix further 
includes a hierarchy of lexical types for creating language-specific 
lexical entries.

All of the constraints in the current matrix are intended to be 
language-independent, and monotonically extensible when developing a
new grammar for a language, though not all types defined in the matrix
will necessarily be employed for any given language.  This strong 
assumption of universality has been of practical utility in the early
development of the matrix, but it has sharply limited the inventory of
constraint definitions that could be supplied to grammar developers
since many generalizations hold only for subsets of languages.  It is
this limitation of the current matrix that we address by introducing
the notion of typology-based modules.

\section{Typology-based modules}


\section{Implementations of three modules}

\section{Management of modules}

\section{Evaluation}

\section*{Acknowledgements}

\section*{References}

%\bibliographystyle{acl}
\bibliographystyle{robbib}
\bibliography{modules}


\end{document}
